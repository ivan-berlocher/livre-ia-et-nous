\chapter*{Avant-propos}
\addcontentsline{toc}{chapter}{Avant-propos}

L'IA générative est arrivée dans nos vies à une vitesse sans précédent. En quelques mois, des centaines de millions de personnes ont commencé à converser avec ChatGPT, Gemini, Claude ou Copilot.

Mais derrière ces interfaces amicales se cachent des questions que personne ne nous a appris à formuler~:

\begin{itemize}
\item \textbf{Quand je parle à ChatGPT, qui m'écoute vraiment~?}
\item \textbf{Mes conversations sont-elles privées~?}
\item \textbf{Pourquoi l'IA me donne parfois des réponses fausses avec tant d'assurance~?}
\item \textbf{En tant qu'Européen, ai-je des droits différents~?}
\end{itemize}

Ce livre n'est ni un tutoriel ni un manifeste technophobe. C'est un guide pour comprendre ce qui se passe --- et ce qui vous concerne.

Il est organisé en \textbf{trois parties} qui peuvent se lire indépendamment~:

\begin{enumerate}
\item \textbf{COMPRENDRE} --- Ce que nous savons avec certitude sur l'IA
\item \textbf{SE PROTÉGER} --- Ce que vous pouvez faire concrètement
\item \textbf{IMAGINER} --- Ce que nous ne savons pas encore, mais pouvons anticiper
\end{enumerate}

Chaque chapitre se termine par des points clés. Les encadrés techniques sont optionnels --- sautez-les si vous voulez aller à l'essentiel.

\begin{quote}
\textit{Ce livre peut se lire linéairement ou par chapitres indépendants.}
\end{quote}

\vspace{1cm}

\section*{À propos de l'auteur}

\textbf{Ivan Berlocher} travaille sur les systèmes intelligents depuis plus de vingt ans.

Ingénieur de formation (DEA STIR, Rennes), il a été Chief Scientist chez Saltlux pendant près de 19 ans, où il a dirigé le centre d'intelligence artificielle et conçu des architectures pour le traitement du langage, les ontologies, et les systèmes de question-réponse. Il a également été CTO chez Goover.ai et Head of Research à l'Open Data Institute (Séoul).

Ses travaux couvrent les moteurs sémantiques, le web de données (projets européens FP7 DaPaaS, Eurostars QAMEL), et plus récemment les architectures pour systèmes agentiques gouvernables.

Il écrit ce livre depuis une conviction simple~: l'IA n'est ni magique ni incompréhensible. Les citoyens ont le droit de savoir comment fonctionnent les systèmes qui influencent leurs vies --- et les moyens de garder le contrôle.

\vspace{0.5cm}
\begin{flushright}
\textit{Séoul -- Lausanne, décembre 2025}
\end{flushright}

\newpage
\tableofcontents
\newpage
